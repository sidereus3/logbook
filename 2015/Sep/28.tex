\section{Sep 28th 2015 - Design Patterns}

Going on studying Design Patterns, yesterday and today I finally got
the following patterns:

\begin{itemize}
\item \textbf{Strategy Pattern}: it allows the developer to separate
  and encapsulate what varies from what remains the same;
\item \textbf{Observer Pattern}: it allows to create a relationship
  between an Observable which send data (push or pull methodology) to
  many Observer, knowing nothing about the Observers. This means that
  there is a loosely coupled design between objects that interact;
\item \textbf{Decorator Pattern}: considering the chance to wrap and
  object with many different type of decorators, this pattern allows
  that opening the main class for extension, but closing it for
  modification;
\item \textbf{Factory Method Pattern}: encapsulate the object creation
  in subclasses, delegating them to decide which class to instantiate;
\item \textbf{Singleton Pattern}: allows the developer having to deal
  with only one istance of that class, providing a global point of
  access to it.
\end{itemize}

I would like to apply the Factory Method Pattern on Marialaura's
components. Specifically, I think this pattern might properly fit the
issue of manage more than one numerical solver, choosing it at run
time.

\par\medskip

Discussed with Marialaura about managing her components in OMS (the
code for computing the Residence Time is defintely too slow):
unfortunately no solution has been found because the problem is due to
the fact that OMS and consequently reader and writer work timestep per
timestep.
