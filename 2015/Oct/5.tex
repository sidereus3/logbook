\section{Oct 5th 2015 - Algorithm}

Discussing with Marialaura about the schematization of the river Adige
and after having read \citeauthor{wang2011:common}
(\citeyear{wang2011:common}) \cite{wang2011:common}, we arrived at the
following ideas:

\begin{itemize}
\item Focus on \textbf{OPN} (Optimal Processor Number): this is going
  to be the innovation of this work;
\item Meteo stations will be treated as input data, at the moment I
  cannot see a way to include them in the tree;
\item The Net (and the binary tree) will be composed by:
  \begin{itemize}
  \item streams;
  \item dams;
  \item lakes;
  \item hydrometers.
  \end{itemize}
\end{itemize}
In order to minimize the users's efforts, just the shapefile of stream
has to be required in order to process the net. The dbf is going to be
read and probably only the following columns are necessary:

\begin{itemize}
\item Hack numbering to get the last stream (to the closure);
\item Coordinates of the starting point of the single stream;
\item Coordinates of the ending point of the single stream.
\end{itemize}
With these informations I should be able to build the tree just
looking at the coordinates of the interconnection points. If a node
has more than two children a ``ghost'' node is introduced, following
\citeauthor{wang2011:common} (\citeyear{wang2011:common})
\cite{wang2011:common}. The operation of building the tree, can be
parallelized, because the numbering of each child depends only on the
number of the parent node. Each branch of the tree is independent.
\par\medskip
The idea is building the tree into an arraylist for example, managing
a key and an object following the \textit{Composite Pattern} which can
be:

\begin{itemize}
\item a complete node;
\item a ghost node;
\item a dam;
\item a leaf.
\end{itemize}
The structure of the Composite Pattern should help in realizing a
flexible framework. To get the data about dams, another dbf could be
read with coordinates of the dams.
\par\medskip
At the moment, no idea about lakes.

\bibliographystyle{apalike} %Stile della bibliografia %
\bibliography{biblio_francesco_serafin}
