%************************************************************************************%
%*          - GrowWorkingHard'script - www.growworkinghard.wordpress.com -          *%
%************************************************************************************%
%                                                                                    %
% This file is a script writed to complete our blog's articles (all the external     %
% reference can be found there). It's needed to explain better concepts and give     %
% to the users an example or a reference.                                            %
% We don't give you the warranty that is the best solution, but we do all the        %
% possible to hit it.                                                                %
%                                                                                    %
% You're invited to try our solution and communicate if something does't work or if  %
% there is a better way to do it!                                                    %
%                                                                                    %
% Follow our blog: <http://www.growworkinghard.wordpress.com>                        %
%                                                                                    %
% Author: Francesco Serafin                                                          %
% Date: 2015-05-01                                                                   %
%                                                                                    %
%************************************************************************************%

% This is a style to exploit for inline code
\def\inline{\lstinline[basicstyle=\ttfamily,keywordstyle={}]}

% This is a LaTeX file to include in the main file
% in order to have several language style available
% to use in listing parts of LaTeX source


\definecolor{lbcolor}{rgb}{0.95,0.95,0.95}
\definecolor{javared}{rgb}{0.6,0,0}             % for strings
\definecolor{javagreen}{rgb}{0.25,0.5,0.35}     % comments
\definecolor{javapurple}{rgb}{0.5,0,0.35}       % keywords
\definecolor{javadocblue}{rgb}{0.25,0.35,0.75}  % javadoc


\lstdefinestyle{numbers}{
        numbers=left,
        stepnumber=1,
        numberstyle=\tiny,
        numbersep=5pt
}

\lstdefinestyle{myFrame}{
        backgroundcolor=\color{lbcolor},
        frame=none
}

%************************************************************************************%
%* B                                                                                *%
%************************************************************************************%

\lstdefinestyle{bashStyle}{
        language=bash,
        style=numbers,
        style=myFrame,
        basicstyle=\footnotesize\ttfamily,
        breaklines=true % sets automatic line breaking
}

%************************************************************************************%
%* C                                                                                *%
%************************************************************************************%

\lstdefinelanguage{cmake}{
        morekeywords={set,project,configure_file,find_package,message,if,else,endif},
        sensitive=false,
        morecomment=[l]{\#},
        morestring=[b]",
}
\lstdefinestyle{cmakeStyle}{
        language=cmake,
        style=numbers,
        style=myFrame,
        basicstyle=\footnotesize\ttfamily,
        % keywordstyle=\color{blue}\ttfamily,
        % stringstyle=\color{red}\ttfamily,
        % commentstyle=\color{javagreen}\ttfamily,
        % morecomment=[l][\color{magenta}]{\#},
        numbers=left,
        numberstyle=\tiny\color{black},
        stepnumber=1,
        numbersep=10pt,
        tabsize=4,
        showspaces=false,
        showstringspaces=false,
        breaklines=true % sets automatic line breaking
}

\lstdefinestyle{cppStyle}{
        language=C++,
        style=numbers,
        style=myFrame,
        basicstyle=\footnotesize\ttfamily,
        keywordstyle=\color{blue}\ttfamily,
        stringstyle=\color{red}\ttfamily,
        commentstyle=\color{javagreen}\ttfamily,
        morecomment=[l][\color{magenta}]{\#},
        numbers=left,
        numberstyle=\tiny\color{black},
        stepnumber=1,
        numbersep=10pt,
        tabsize=4,
        showspaces=false,
        showstringspaces=false,
        breaklines=true % sets automatic line breaking
}

%************************************************************************************%
%* J                                                                                *%
%************************************************************************************%

\lstdefinestyle{javaStyle}{
        language=Java,
        style=numbers,
        style=myFrame,
        keywordstyle=\color{javapurple}\bfseries,
        basicstyle=\footnotesize\ttfamily,
        stringstyle=\color{javared},
        commentstyle=\color{javagreen},
        morecomment=[s][\color{javadocblue}]{/**}{*/},
        numbers=left,
        numberstyle=\tiny\color{black},
        stepnumber=1,
        numbersep=10pt,
        tabsize=4,
        showspaces=false,
        showstringspaces=false,
        breaklines=true % sets automatic line breaking
}

%************************************************************************************%
%* R                                                                                *%
%************************************************************************************%

\lstdefinestyle{Rstyle}{ %
        language=R,                     % the language of the code
        basicstyle=\footnotesize\ttfamily,       % the size of the fonts that are used for the code
        numbers=left,                   % where to put the line-numbers
        numberstyle=\tiny\color{gray},  % the style that is used for the line-numbers
        stepnumber=1,                   % the step between two line-numbers. If it's 1, each line
        % will be numbered
        numbersep=5pt,                  % how far the line-numbers are from the code
        backgroundcolor=\color{white},  % choose the background color. You must add \usepackage{color}
        showspaces=false,               % show spaces adding particular underscores
        showstringspaces=false,         % underline spaces within strings
        showtabs=false,                 % show tabs within strings adding particular underscores
        frame=single,                   % adds a frame around the code
        rulecolor=\color{black},        % if not set, the frame-color may be changed on line-breaks within not-black text (e.g. commens (green here))
        tabsize=2,                      % sets default tabsize to 2 spaces
        captionpos=b,                   % sets the caption-position to bottom
        breaklines=true,                % sets automatic line breaking
        breakatwhitespace=false,        % sets if automatic breaks should only happen at whitespace
        title=\lstname,                 % show the filename of files included with \lstinputlisting;
        % also try caption instead of title
        keywordstyle=\color{blue},      % keyword style
        commentstyle=\color{dkgreen},   % comment style
        stringstyle=\color{mauve},      % string literal style
        escapeinside={\%*}{*)},         % if you want to add a comment within your code
        morekeywords={*,...}            % if you want to add more keywords to the set
}
