\section{Sep 26th 2016 - Back to life}

Back writing after some time.

Today's work: well, I started to take classes of Algorithms and Data
Structures at DISI. The purpose is to have a background before taking
classes at the CS Department at CSU.\par\medskip

In the meanwhile, my main goal is the implementation of a Tree data
structure inside OMS core. Olaf allowed me to push directly on a
forked version of the OMS core on
\url{http://alm.engr.colostate.edu}. The starting point of the
analisys is the \inline!.sim! file to understand how to connect the
OMS core to the user experience. A new thread is created at row 770 in
\inline!SimPanel.java! after clicking the \inline!runButton!, thus the
parsing of the \inline!.sim! file should be done more or less in this
part of the source code. A \inline!ProcessExcecution! starts at line
783, calling the method \inline!exec()! at row 205 in
\inline!ProcessExecution.java! class.

Furthermore, I completed the work on the building script of the
\url{https://github.com/geoframecomponents/GEOframeUtils}
repository. The building system is
Gradle\thanks{\includegraphics{gradle.png}} and the dependencies
management is automatically done through maven repositories. For this
reason, a \textbf{maven repo should be organized} to put the
\inline!jars! of each component for who is going to use them as
API.\par\medskip

\textbf{FIXED UGLY FONT RENDERING} in Netbeans on ArchLinux. The trick
is running \inline!Netbeans! with the following options

\begin{lstlisting}[style=bashStyle]
  netbeans -J-Dswing.aatext=true -J-Dawt.useSystemAAFontSettings=on
\end{lstlisting}%
